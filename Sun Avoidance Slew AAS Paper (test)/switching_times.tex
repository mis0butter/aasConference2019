\documentclass[letterpaper, preprint, paper,11pt]{AAS}	% for preprint proceedings

\usepackage{bm}
\usepackage{amsmath}
\usepackage{subfigure}
%\usepackage[notref,notcite]{showkeys}  % use this to temporarily show labels
\usepackage[colorlinks=true, pdfstartview=FitV, linkcolor=black, citecolor= black, urlcolor= black]{hyperref}
\usepackage{overcite}
\usepackage{footnpag}			      	% make footnote symbols restart on each page

% Added packages 2019 Feb 17
\usepackage{float}
\usepackage{amssymb}

\begin{document}
	

%%%%%%%%%%%%%%%%%%%%%% copy and paste below: 

\begin{itemize}
	\item For a given $\ddot{\phi}_{max}$ and $\dot{\phi}_{max}$, there is a threshold slew angle $\phi_T$ that determines whether the slew will have a period of coasting. The threshold slew angle is found by: 
		\begin{equation}
		\phi_T = \frac{(\dot{\phi}_{max})^2}{\ddot{\phi}_{max}}
		\end{equation}
%	\item The switching times $t_1$, $t_2$, and final time $t_f$ are calculated differently depending on whether the slew angle $\phi_f$ is greater or less than $\phi_T$.  	
	\item Using the conditions, $\dot{\phi}(t_1)=\dot{\phi}_{max}$, $\dot{\phi}(t_f)=\dot{\phi}_f$, $\phi(t_f)=\phi_f$, the switching times can be determined as follows. 
	\item If $\phi_f < \phi_T$, the switching times $t_1$ and final time $t_f$ are determined thus:
\end{itemize}
\begin{equation}
	t_f = \sqrt{\frac{4\phi_f}{\ddot{\phi}_{max}}}
\end{equation} 
\begin{equation}
	t_1 = \frac{t_f}{2}
\end{equation}

\begin{itemize}
	\item If $\phi_f > \phi_T$, then the slew will have a period of coasting with constant $\dot{\phi}(t)$. The switching times $t_1$, $t_2$, and final time $t_f$ are then calculated as: 
\end{itemize}

% Set of equations that determine switching times if phi_f > phi_T 
\begin{equation}\label{t1cons}
	t_1=t_0+\frac{\dot{\phi}_{max}-\dot{\phi}_0}{\ddot{\phi}_{max}},
\end{equation}
\begin{equation}\label{t2cons}
	\begin{split}
	t_2 = & t_1 + \frac{1}{\dot{\phi}_{max}}\Big[ \phi_f-\dot{\phi}_0(t_1-t_0)-\frac{1}{2}\ddot{\phi}_{max}(t_1-t_0)^2 \\
	& -\frac{\dot{\phi}_{max}(\dot{\phi}_{max}-\dot{\phi}_f)}{\ddot{\phi}_{max}}+\frac{(\dot{\phi}_{max}-\dot{\phi}_f)^2}{2\ddot{\phi}_{max}} \Big],
	\end{split}
\end{equation}
\begin{equation}\label{tfcons}
	t_f = t_2 - \frac{\dot{\phi}_f - \dot{\phi}_{max}}{\ddot{\phi}_{max}}
\end{equation}

%%%%%%%%%%%%%%%%%%%%%% copy and paste above 

\end{document}



% old t_f 
% 
%\begin{equation}\label{tfcons}
%	t_f=t_1+\frac{1}{\dot{\phi}_{max}}\Big[ \phi_f-\dot{\phi}_0(t_1-t_0)-\frac{1}{2}\ddot{\phi}_{max}(t_1-t_0)^2+\frac{(\dot{\phi}_{max}-\dot{\phi}_f)^2}{2\ddot{\phi}_{max}} \Big].
%\end{equation} 

